\chapter{Заключение}

В ходе моей работы над проектом автоматизированной сборки модулей CMS трекера второй стадии модернизации были достигнуты следующие результаты:

\begin{itemize}
\item Внесены существенные модификации в приложение PSAuto, включая:
\begin{itemize}
\item Интегрирован модуль управления преобразованием изображений, получаемых от камеры, что позволяет улучшить качество работы алгоритма распознавания образов.
\item Интегрирован гибкий и удобный метод управления системой отрицательного давления.
\item Исходный код приложения модифицирован таким образом, чтобы удовлетворять шаблону проектирования MVC – Model View Controller.
\item Введены некоторые методы защиты приложения и системы от ошибок, в том числе и пользователя.
\item Интегрирован и протестирован модуль приложения, ответственный за первый прототип полностью автоматизированной сборки.
\end{itemize}
\item Доказана возможность автоматизированной сборки в рамках указанных требований.
\item Собран первый прототип модуля, показывающий потенциал системы.
\item Была разработана, изготовлена, внедрена в систему и успешно протестирована сборочная платформа.
\item Была исследована и успешно применена технология применения быстросохнущего вещества в процессе сборки.
\item Было проведено множество тестов всей системы демонстрирующих возможности системы и предоставляемую ею точность сборки.
\end{itemize}

Ссылка на репозиторий с кодом приложения: https://github.com/bryksa/cmstkmodlab.


%\section{Results}


%\section{Future plans}