\chapter{Сборка модулей}

Сборку высокоточных детектирующих модулей такого рода обычно проводят вручную, используя изготовленные на заказ специальные механические фиксирующие платформы. В качестве альтернативы данному привычному методу предлагается использовать систему автоматизированной сборки.

\section{Ручная сборка с помощью механической фиксирующей платформы}

Один из способов сборки модулей -- ручная сборка с помощью изготовденной на заказ специальной механической фиксирующей платформы, прототип которой показан на Рисунке \ref{fig:mechanical_jig}) \cite{Automated_assembly_slides}.

\begin{figure}[ht]\centering
\includegraphics[width=0.7\linewidth]{Data/Module_assembly/Mechanical_jig.png}
\caption{Прототип механической фиксирующей платформы.}
\label{fig:mechanical_jig}
\end{figure}

При таком способе сборке все составные части размещаются на платформе и собираются вручную. Для обеспечения необходимой точности сборки фиксирующая платформа имеет три опорных стопора: один на длинной стороне и два вдоль более короткой стороны. С противоположных стопорам сторон собираемый модуль аккуратно прижимается специальными пружинами. Вместе они обеспечивают необходимую точность позиционирования компонентов модуля перед и во время сборки/склеивания.

Однако, описанный ваше способ сборки обладает рядом недостатков. Для сборки с его помощью необходимо много времени и данная механическая система плохо масштабируется на промышленные масштабы сборки. Также данный метод обладает относительно плохой повторяемостью собранных модулей, пусть и в пределах необходимой точности, и не предоставляет возможностей контроля процесса сборки. Более того, фиксирующая платформа выставляет высокие требования качество её изготовления (точность выточки деталей должна составлять несколько микрон), а также стопоры требуют регулярной калибровки из-за возможных сдвигов и/или стачиваний их поверхностей. Наконец, данная механическая система требует максимум участия человека в её работе. Несмотря на расчётную точность системы в теории, факт участия человека в сборке означает, что всегда будет существовать какой-то процент бракованных модулей, произведённых только по вине человека.

\section{Система автоматизированной сборки}

Предлагаемая система автоматизированной сборки состоит из трёх подсистем: двигательной подсистемы, подсистема наблюдения, подсистема отрицательного давления (Figure \ref{fig:auto_assembly_system}) \cite{AutomatedAssembly_tutorial}.

\begin{figure}[ht]\centering
\includegraphics[width=1\linewidth]{Data/Module_assembly/Automated_assembly_system.png}
\caption{Предлагаемая система автоматизированной сборки.}
\label{fig:auto_assembly_system}
\end{figure}

\emph{Двигательная подсистема} обеспечивает точные передвижения необходимые для расположения компонентов, из которых состоит СРСП. Движение обеспечивается благодаря двум подвижным элементам двигательной подсистемы, обеспечивающим, первый, по трём координатам и, второй, вращение. Специально изготовленное приспособление из алюминия, далее именуемое "роборукой", смонтированной на тот элемент двигательной подсистемы, который обеспечивает движение по трём координатам, таким образом позволяя перемещать компоненты модуля в прямоугольных координатах. Они размещаются на вращательной установке, осуществляющей вращение в  плоскости xOy той же прямоугольной системы координат. Элементы двигательной подсистемы управляются специальным блоком управления. Вся аппаратная часть двигательной подсистемы произведена фирмой Lang, которая точность перемещений до 4~мкм и 2~мрад соответственно. \emph{Подсистема отрицательного давления} позволяет закреплять компоненты собираемого модуля на роборуке и вращательной установке. Она состоит из одного насоса, подводящей отрицательное давление к четырём переключаемым клапанам, которые в свою очередь распределяют отрицательное давление по четырём независимым линиям. Переключение клапанов осуществляется путём приложения управляющих сигналов напряжением в 12~В (12~В -- включенное состояние, 0~В -- выключенное). Управляющие сигналы обеспечиваются специальной релейной картой. Одна линия отрицательного давления подведена к устройству захвата компонентов модуля, крепящегося на конце роборуки, другие -- к сборочной платформе. Захватное устройство сделано из ESD пластика, в котором располагается внутренняя камера для распределения подключённого отрицательного давления по специальным направленным вниз присоскам, которые слегка выступают за нижнюю грань захватного устройства. Схематичное изображение устройства захвата отрицательным давлением показано на Рисунке \ref{fig:pick_up_tool}.

\begin{figure}[ht]\centering
\includegraphics[width=0.7\linewidth]{Data/Module_assembly/Pick_up_tool.png}
\caption{Устройства захвата системы автоматизированной сборки.}
\label{fig:pick_up_tool}
\end{figure}

Процедура захвата происходит путём касания присосками сенсора, включения подачи отрицательного давления устройству захвата и непосредственному перемещению роборуки вверх с захваченным сенсором. Процедура высвобождения сенсора начинается с контакта нижней поверхности прикреплённого сенсора (или опорной пластины) поверхности на которую необходимо поместить прикреплённый компонент, выключения подачи отрицательного давления и отвода роборуки в сторону. С целью избежания нежелательных перемещений компонента модуля во время увода роборуки, на сборочной платформе компонент фиксируется специальными присосками снизу, принцип работы которых полностью схож с фиксацией на устройстве захвата.

\emph{Подсистема наблюдения} делает снимки компонентов, проводит распознавание их расположения и ориентации в пространстве, что является особенно важным для осуществления точной сборки модуля. Она представлена камерой высокого разрешения фирмы IDS и крепится к роборуке. Она зафиксирована в положении, направленном вниз. Основной задачей камеры является делать снимки компонентов до и после их захвата и опускания вниз для последующей обработки этих снимков специализированным программным обеспечением, которое точно определяет положение компонента и его ориентацию.

\section{Сборочная платформа}

Одной из наиболее важных частей системы автоматизированной сборки является сборочная платформа. Её основной задачей является фиксация модулей во время сборки модуля.

Сборочная платформа должна выполнять следующие требования:

\begin{enumerate}
\setlength\itemsep{-0.5em}
\item Фиксация всех необходимых компонентов с помощью отрицательного давления.
\item Обеспечивать возможность расположения двух распорок модуля строго параллельно и на определённом расстоянии.
\item Быть по возможности лёгкой и иметь центр масс как можно ближе к оси вращения вращательной установки, на которой платформа будет установлена.
\end{enumerate}

Чтобы выполнить упомянутые выше требования, был предложен следующий дизайн (Рисунок \ref{fig:platform_design}).

\begin{figure}[ht]\centering
\includegraphics[width=0.7\linewidth]{Data/Module_assembly/Platform_design.png}
\caption{Эскиз сборочной платформы.}
\label{fig:platform_design}
\end{figure}

Сборочная платформа имеет два углубления, в каждом из которых находятся по три опорных стопора обеспечивающих точное позиционирование распорок модуля. В сравнении с механической фиксирующей платформой, стопоры данной сборочной платформы являются частью платформы, а не прикручиваются отдельно, таким образом они не нуждаются регулярной в калибровке. Однако, наличие данных углублений делает невозможным размещение других компонентов модуля (сенсоров опорной пластины) непосредственно поверх них, так как слишком большая часть их площади остаётся без поддержки под давлением устройства захвата сверху во время склеивания компонентов модуля. Данная проблема была решена путём размещением плоских компонентов модуля (сенсор, опорная пластина) перпендикулярно распоркам на платформе. В данном случае только небольшая часть данных компонентов не будет иметь поддержки снизу, что не является критичным. Перпендикулярное размещение некоторых компонентов собираемого модуля может быть с лёгкостью осуществлено вращательной платформой, на которой находится сборочная платформа. Центр масс спроектированной платформы находится очень близко к оси вращения -- меньше 1~мм. В то время как её масса составляет около 1~кг, таким образом никаких нежелательных эффектов влияющих на точность работы вращательной платформы не ожидается.

Внутри сборочной платформы располагаются две независимые линии отрицательного давления: первая -- для удержания распорок, вторая -- для удержания сенсоров и опорной пластины. Линия распорок распределяет давление по ряду отверстий (0.7~мм в диаметре) на дне углублений для распорок. Размер и расположение данных отверстий определяется формой распорки (Рисунок \ref{fig:al_spacer}). Эти отверстия не оборудованы присосками из-за своего очень малого размера. Присосок таких размеров попросту нету на рынке. Однако, соприкасающиеся поверхности (дно углублений платформы и соответствующая граня распорки) имеют достаточную плоскость, чтобы утечка "вакуума" была в допустимых пределах и распорки были зафиксированы достаточно плотно.

\begin{figure}[ht]\centering
\includegraphics[width=0.7\linewidth]{Data/Module_assembly/Al_spacer.png}
\caption{Алюминиевый прототип распорки для PS модуля.}
\label{fig:al_spacer}
\end{figure}

Вторая линия отрицательного давления распределяет его между присосками, удерживающих сенсоры или опорную пластину. В сборочной платформе используются присоски, абсолютно идентичные захватывающему устройству. Так же, как и в захватывающем устройстве, они слегка выступают за поверхность сборочной платформы. Однако, они не должны препятствовать процессу сборки, а именно этапу приклеивания сенсора к распоркам (этапы сборки будут подробно описаны позднее в данной Главе). Именно поэтому высота углублений меньше толщины распорок, что позволяет верней части распорок быть выше верхней части присоски. Другими словами, присоски не препятствуют этап сборки, в которых не участвуют, так как находятся ниже верхней части распорок.

\section{Быстросохнущее вещество склеивания.}

Другой важной частью системы автоматической сборки является склеивающее вещество, используемое во время сборки, говоря точнее -- время его высыхания. Другими словами, мало смысла оставлять один модуль на длительное время зафиксированным в установке, просто ожидая пока высохнет склеивающее вещество. Например, основное клейкое вещество, используемое в сборке модуля, требует 24~часов для высыхания. Именно поэтому система автоматизированной сборки нуждается в технологии, которая позволит избежать длительного ожидания. Одно из предложенных решений -- использование небольшого количества быстросохнущего клейкого вещества в дополнении к основному. Данное вещество должно удовлетворять следующим требованиям:

\begin{enumerate}
\setlength\itemsep{-0.5em}
\item Обеспечивать необходимую прочность соединения спустя 15 минут применения.
\item Не вступать во взаимодействие с основным клеящим веществом.
\item Обеспечивать тонкий слой (до 30~мкм).
\end{enumerate}

Более того, упомянутые выше требования должны быть выполнены с использованием как можно меньшего количества вещества. Для тестов использовались небольшие алюминиевые бруски формы прямоугольного параллелепипеда (в качестве приближения распорок и усиленного карбоновым волокном алюминия) и образцы стела (в качестве приближения кремневых сенсоров). Достижение перечисленных требований в высокой степени зависит от двух факторов: первый -- способ применения клеящего вещества, второй -- непосредственный свойства вещества.

Возможны множество способов применения быстросохнущего клеящего вещества, среди них: несколько его капель внутри слоя основного клеящего вещества, несколько его капель по краям соединения, заполнение скоса распорок быстросохнущим клеящим веществом, несколько его капель поверх края соединения и др. Для описываемой системы было принято проверить заполнение скоса стопоров быстросохнущим клеящим веществом и несколько его капель по краям соединения. Несмотря на то, что проклеивания по скосу может обеспечить необходимую прочность соединения после 15~минут, намного практичнее применять несколько капель вещества по краям соединения. Оно применялось способом, показанным на Рисунке \ref{fig:glue_application}.

\begin{figure}[ht]\centering
\includegraphics[width=0.7\linewidth]{Data/Module_assembly/Fast_glue_application.png}
\caption{Способ применение быстросохнущего вещества во время тестов. На практике достаточно капель намного меньших размеров для обеспечение необходимой прочности соединения.}
\label{fig:glue_application}
\end{figure}

Вторая часть вопроса -- поиск подходящего быстросохнущего вещества на рынке. Основная сложность вопроса состоит в комбинации двух свойств: \emph{быстрое высыхание} и \emph{низкая вязкость}. Низкая вязкость есть неотъемлемой свойство клеящего вещества, если необходимо обеспечить тонкий слой вещества. Однако, быстрое высыхание означает, что вещество должно стать твёрдым быстро, чего в свою очередь намного проще достичь, если вещество изначально будет большей вязкости. Несмотря на данные сложности, были подобраны несколько кандидатов для тестирования:

\begin{enumerate}
\setlength\itemsep{-0.5em}
\item \textit{Polytec EP 660}. Согласно технической документации \cite{Polytec_EP_660_datasheet}, время его высыхания составляет до 16~часов, что выходит далеко за рамки требований. Однако, производитель отметил, что необходимая прочность соединения может быть достигнута без полного высыхания. К сожалению, данное вещество, ожидаемо, не обеспечило необходимое соединение по прошествии 15~минут. Единственным его преимуществом является факт того, что основное клеящее вещество было изготовлено тем же производителем.
\item \textit{Loxeal 31-42}. Согласно технической документации \cite{Polytec_EP_660_datasheet}, время его полного высыхания составляет $\approx20-30$~минут, в то время как время "схватывания" составляет около 3-8 минут. В результате после 15 минут данное вещество обеспечивает необходимую прочность соединения.
\item \textit{Wekem WK5}. Согласно технической документации \cite{Wekem_WK_5_datasheet}, время его высыхания составляет около пяти минут. Данный образец также обспечивает необходимое качество соединения после 15~минут высыхания, но клеящее вещество от Loxeal имеет лучшее качество и удобнее в применении. Более того, Loxeal 31-42 обеспечивает более тонкий слой при высыхании под одинаковым давлением -- $<~20~мкм$ -- в то время как Wekem --   $\approx40~um$ (Рисунок \ref{fig:glue_thickness}).
\end{enumerate}

\begin{figure}[ht]\centering
\includegraphics[width=0.7\linewidth]{Data/Module_assembly/Loxeal_393g_point_1.png}
\caption{Толщина слоя быстросохнущего клеящего вещества Loxeal~31-42 при высыхании под давлением $\approx20~g/cm^{2}$}
\label{fig:glue_thickness}
\end{figure}

Подводя итоги, Loxeal 31-42 показал лучшие результаты, удовлетворив все требования: он обеспечивает необходимую прочность соединения после 15~минут высыхания под небольшим давлением (around $20~g/cm^{2}$), создавая при этом тонкий слой -- $<~20~мкм$, и демонстрирует отсутствие взаимодействия с основным клеящим веществом.

\section{Процесс автоматизированной сборки}

Процесс автоматизированной сборки СРСП для PS модуля состоит из следующих шагов:

\begin{enumerate}
\setlength\itemsep{-0.5em}
\item \emph{Подготовка верхнего сенсора.} Сперва необходимо поместить верхний сенсор на сборочную платформу и зафиксировать отрицательным давлением. После распознать его позицию и ориентацию в пространстве с помощью камеры и специального программного обеспечения и сохранить данную информацию. Затем захватить и поднять его с помощью устройства захвата и оставить прикреплённым. Начиная с самого первого шага подача отрицательного давления на устройство захвата не прекращается в течении всего процесса сборки.
\item \emph{Подготовка распорок.} Как только верхний сенсор прикреплён к устройству захвата и поднят с платформы, сборочная платформа должна повернуться на 90~градусов таким образом, что распорки будут ложиться в специально отведённый для них углубления корректно ориентированными относительно верхнего сенсора. Далее их необходимо аккуратно прислонить к опорным стопорам и включить подачу отрицательного давления для их фиксации. После система должна найти и определить положение опорного маркера на сборочной платформе, что позволит однозначно рассчитать положение распорок в пространстве. Отрегулировать их ориентацию с помощью вращательной установки таким образом, чтобы они были параллельны оси Ox двигательной установки, что означает быть сонаправленным верхнему сенсору.
\item \emph{Склеивание верхнего сенсора с распорками.} Как только верхний сенсор и распорки готовы, их положения и ориентации определены, программное обеспечение может вычислить путь для устройства захвата с удерживаемым им верхним сенсором к распоркам. После необходимо поместить основное клеящее вещество на распорки и несколько капель быстросохнущего вещества на краях распорок и передвинуть устройство захвата в расчётную позицию для склеивания с распорками. Особое внимание стоит уделить движению устройства захвата по Z координате, так как это напрямую влияет на толщину слоя клеящего вещества.
\item \emph{Подготовить нижний сенсор.} По истечении 15~минут капли быстросохнущего вещества обеспечивают необходимое соединение для извлечение склеенных распорок и верхнего сенсора. Для чтобы это сделать необходимо сначала отключить подачу отрицательного давления, удерживающего распорки, и после просто поднять устройство захвата, которое уже будет удерживать не только верхний сенсор, но и приклеенные к нему распорки. После того, как платформа очищена её следует повернуть на 90~градусов так, что нижний сенсор может быть помещён на платформу и зафиксирован отрицательным давлением. Далее необходимо так же, как и для верхнего сенсора, определить его положение и ориентацию с помощью камеры и специального программного обеспечения и сохранить данную информацию. Отрегулировать ориентацию сенсора с помощью вращательной установки таким образом, чтобы он был параллелен оси Ox двигательной установки, что означает быть сонаправленным верхнему сенсору и распоркам.
\item \emph{Склеивание нижнего сенсора с распорки~+~верхний сенсор.} Все необходимые для данного этапа компоненты зафиксированы и их расположение известно. Следовательно, программное обеспечение способно рассчитать путь для устройства захвата и переместиться в позицию склеивания после нанесения главного и быстрсохнущего клеящего вещества.
\item \emph{Подготовка опорной пластины.} По прошествии 15~минут склеенная структура может быть поднята. Далее необходимо поместить опорную пластину на сборочную платформу согласно трём опорным штифтам и отверстий для них в самой пластине и зафиксировать её с помощью отрицательного давления. Данные штифты не представлены в текущей версии сборочной платформы.
\item \emph{Склеивание сенсор-распорки-сенсор структуры с опорной пластиной.} Так как программное обеспечение уже знает расположения опорного маркера на сборочной платформе, следовательно оно способно вычислить позицию склеивания для текущего этапа. Далее устройство захвата передвигается в эту позицию после нанесения основного и быстросохнущего клеящего вещества.
\item \emph{Автоматическая сборка завершена.} По прошествии 15~минут быстросохнущее вещество обеспечивает необходимую прочность соединение и собранный СРСП может быть убран со сборочной платформы и оставлен в другом месте до полного высыхания склеивающих веществ на 24~часа. Однако опорные штифты опорной пластины очень плотно подходят самой пластине, следовательно ручное снятие СРСП может вероятно вызвать его разрушение. Поэтому лучше доверить и данный этап двигательной установке, которая может безопасно поднять собранный СРСП вертикально вверх, сняв его с опорных штифтов.
\end{enumerate}