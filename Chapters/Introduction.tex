\chapter{Введение}

Большой адронный коллайдер (БАК) -- крупнейший ускоритель частиц в мире. Во время его первого периода работы с 2010 по 2013 с его помощью были сделаны выдающиеся достижения. Возможно одно из наиболее известных -- открытие теоретически предсказанного бозона Хиггса, за что Франсуа Энглер (François Englert) and Питер Хиггс (Peter Higgs) была присуждена Нобелевская премия по физике в 2013 году. Исследование явлений при столкновении частиц высоких энергий БАК происходит в четырёх позициях, расположенных вдоль основного кольца ускорителя и соответствующих четырём детекторам частиц -- ATLAS, CMS, ALICE и LHCb \cite{ref_cern_home}.


\section{Компактный мюонный соленоид}

Компактный мюонных соленоид (Compact Muon Solenoid или CMS) -- это цилиндрический детектор элементарных частиц, разработанный для измерения широкого спектра параметров различных элементарных частиц, производимых во время столкновений (коллизий) протонов и/или тяжёлых ионов БАК. Детектор имеет длину около 28~метров и диаметр около 15~метров. Он является самым тяжёлым детектором элементарных частиц в мире и весит приблизительно 14000~тонн. Своё имя CMS детектор получил благодаря его 3-м ключевым характеристикам: его относительно компактный размер, его исключительная способность в детектировании и измерении мюонов и третья, его основная особенность, -- сверхпроводящий соленоид, способный создать магнитное поле до 3.8~Т.

\begin{figure}[ht]\centering
\includegraphics[width=0.9\linewidth]{Data/Introduction/CMS_layers.png}
\caption{Вид на слои CMS детектора в поперечном разрезе.}
\label{fig:cms_layers}
\end{figure}

CMS детектор состоит из нескольких слоёв, каждый из которых выполняет свою отдельную роль в детектировании и измерении пролетающих сквозь них элементарны частиц. Схематическое изображение поперечного сечения данных слоёв и их задачи в восстановлении пути элементарной частицы показаны на Рисунке \ref{fig:cms_layers}.

\section{Вторая стадия модернизации CMS трекера}

После проведения второй стадии модернизации БАК будет обладать гораздо большей светимостью. Данный режим его работы получил название High Luminosity LHC (HL-LHC) (с англ. -- БАК на большой светимости). Серьёзной сложностью условий работы данного режима БАК является является намного больший поток производимых данных, который многократно превосходит пропускную способность электронных устройств ввода-вывода. Однако подавляющее большинство элементарных частиц, образованных в результате работы HL-LHC малоинтересны в текущих исследованиях физики высоких энергий и характеризуются низким поперечным импульса. Таким образом игнорируя попадания элементарных частиц с низким поперечным импульсом, можно значительно уменьшить объём считываемых данных. С целью обеспечения возможности распознавания момента импульса на уровне аппаратного обеспечения был создан проект двухслойных детектирующих модулей. Основная идея работы новых модулей состоит в разграничении проходящих частиц с низким и высоким поперечным импульсом основываясь на приближенном измерении кривизны траектории вызванной магнитным полем, в котором находится модуль. Например, частица с высоким поперечным импульсом после прохождения детектирующего пикселя/полоски первого сенсора пройдёт через соответствующий пиксель/полоску соседнего второго сенсора. В то время как частица с низким поперечным импульсом будет иметь большую кривизну траектории и пройдёт немного в стороне от соответствующего элемента первого сенсора на втором сенсоре. Варьируя расстояние между сенсорами и количество соседствующих детектирующих пикселей/полосок необходимых для сопоставления областей прохождения на смежных сенсорах модуля (2 соседних детектирующих полоски на Рисунке \ref{fig:low_high_pT}), возможно выставить пороговое значение поперечного импульса для проходящей частицы~\cite{CMS_TECH_PHASE_II}.


\begin{figure}[ht]\centering
\includegraphics[width=0.8\linewidth]{Data/Introduction/Low_high_pT.png}
\caption{Пример разграничения высокого и низкого поперечного импульса. Частицы, которые пройдут в пределах двух соседних детектирующих полосок второго сенсора от соответствующий полоски первого сенсора, будут считаться как частицы с высоким поперечным импульсом.}
\label{fig:low_high_pT}
\end{figure}


\textbf{Двухслойный детектирующий модули}

Трекер CMS детектора второй стадии модернизации будет использовать два типа детектирующих модулей -- 2S модули и PS модули. Чтобы достигнуть эффективного распознавания частиц с низким поперечным импульсом по всему объёму трекера, в различных его частях сенсоры модулей будут располагаться на разном расстоянии друг от друга. Для 2S (PS) модулей расстояние будет варьироваться между 1.8 и 4~мм (1.6, 2.6 - 4~мм). PS модуль в разборном виде показан на Рисунке \ref{fig:ps_exploaded}.

\begin{figure}[ht]\centering
\includegraphics[width=0.8\linewidth]{Data/Introduction/PS_exploaded.png}
\caption{PS модуль в разборном виде.}
\label{fig:ps_exploaded}
\end{figure}

В PS модуле сенсоры приклеены к распоркам изготовленных из алюминия, усиленного углеродным волокном. Они обеспечивают необходимую теплопроводность, остро необходимую для охлаждения модуля. Два сенсора вместе с распорками в свою очередь приклеены к опорной пластине из углеродного волокна. Эта структура впоследствии будет называться Сенсор-Распорка-Сенсор-Пластина (СРСП). Данная работа ориентирована исключительно на сборку 
СРСО. Требуемые допущения по точности сборки СРСП показаны на Рисунке \ref{fig:ps(2s)_precision}. Для PS модуля сенсоры должны быть расположены с относительной точностью в 40~мкм измеряя с короткой стороны. Данная точность соответствует максимальной неточности в относительном вращении сенсоров на величину 0.8~мрад \cite{AutomatedAssembly_tutorial}.

\begin{figure}[ht]\centering
\includegraphics[width=0.4\linewidth]{Data/Introduction/PS(2S)_precision.png}
\caption{Требования точности к сборке PS и 2S модулей.}
\label{fig:ps(2s)_precision}
\end{figure}
