\chapter{Введение}

Большой адронный коллайдер (БАК) -- крупнейший ускоритель частиц в мире. Во время его первого периода работы с 2010 по 2013 с его помощью были сделаны выдающиеся достижения. Возможно одно из наиболее известных -- открытие теоретически предсказанного бозона Хиггса, за что Франсуа Энглер (François Englert) and Питер Хиггс (Peter Higgs) была присуждена Нобелевская премия по физике в 2013 году. Исследование явлений при столкновении частиц высоких энергий БАК происходит в четырёх позициях, расположенных вдоль основного кольца ускорителя и соответствующих четырём детекторам частиц -- ATLAS, CMS, ALICE и LHCb \cite{ref_cern_home}.


\section{Компактный мюонный соленоид}

Компактный мюонных соленоид (Compact Muon Solenoid или CMS) -- это цилиндрический детектор элементарных частиц, разработанный для измерения широкого спектра параметров различных элементарных частиц, производимых во время столкновений (коллизий) протонов и/или тяжёлых ионов БАК. Детектор имеет длину около 28~метров и диаметр около 15~метров. Он является самым тяжёлым детектором элементарных частиц в мире и весит приблизительно 14000~тонн. Своё имя CMS детектор получил благодаря его 3-м ключевым характеристикам: его относительно компактный размер, его исключительная способность в детектировании и измерении мюонов и третья, его основная особенность, -- сверхпроводящий соленоид, способный создать магнитное поле до 3.8~Т.

\begin{figure}[ht]\centering
\includegraphics[width=0.9\linewidth]{Data/Introduction/CMS_layers.png}
\caption{Вид на слои CMS детектора в поперечном разрезе.}
\label{fig:cms_layers}
\end{figure}

CMS детектор состоит из нескольких слоёв, каждый из которых выполняет свою отдельную роль в детектировании и измерении пролетающих сквозь них элементарны частиц. Схематическое изображение поперечного сечения данных слоёв и их задачи в восстановлении пути элементарной частицы показаны на Рисунке \ref{fig:cms_layers}.

\section{Вторая стадия модернизации CMS трекера}

After Phase II Upgrade, the LHC will provide a much higher luminosity. This regime is known as the High Luminosity LHC (HL-LHC). A serious problem presented by these conditions is the enormous data readout rates that exceed far beyond the bandwidth foreseen for the readout electronics.
However, the vast majority particles produced in the HL-LHC conditions are not of direct interest for new physics searches and are characterized by low transverse momentum. Thus rejecting tracker “hits” related to low transverse momentum particles can significantly reduce the amount of data to be readout. In order to provide momentum discrimination at the hardware level, a 2-layer module design was created. The central idea of the new modules is to provide fast discrimination between low and high transverse momentum particles by estimating the track curvature caused by the magnetic field within the volume of the module itself. For example, particle with high transverse momentum after hitting some pixel/strip at the first sensor layer would hit on of neighboring pixels/strips of the respective pixel/strip on the second layer. While a particle with low transverse momentum would have a more curved trajectory an hit pixel/strips at a displaced position from the first hit. By varying the distance between sensors and number of neighboring pixel/strips required to match hits in adjacent sensors, (2 neighboring strips in the Figure \ref{fig:low_high_pT}) it is possible to set the transverse momentum threshold for a hit~\cite{CMS_TECH_PHASE_II}.


\begin{figure}[ht]\centering
\includegraphics[width=0.8\linewidth]{Data/Introduction/Low_high_pT.png}
\caption{An example of distinguishing high and low transverse momentum. Particles, which hit any of two neighboring strips or the respective strip itself, would be record as high transverse momentum particles.}
\label{fig:low_high_pT}
\end{figure}


\textbf{Two layer Modules}

The CMS Phase-II Tracker will utilize two types of modules, 2S modules and PS modules. To achieve efficient rejection of low-pT (low transverse momentum) particles throughout the Tracker volume, modules in different regions will make use of a few different sensor spacings. For 2S (PS) modules, spacings of 1.8 and 4 mm (1.6, 2.6 and 4 mm) are foreseen. These modules will be used in the end-cap disks as well as the central barrel region of the Tracker. An exploded view of a PS module is shown in Figure \ref{fig:ps_exploaded}.
\begin{figure}[ht]\centering
\includegraphics[width=0.8\linewidth]{Data/Introduction/PS_exploaded.png}
\caption{Exploaded view of Pixesl Sensor Module.}
\label{fig:ps_exploaded}
\end{figure}

In the PS module, the sensors are glued to a carbon-fibre reinforced Aluminium (AL-CF) spacers which act as spacers and provide the thermal conductance crucial for the cooling of the module. The two sensors and spacers are in turn glued the carbon-fibre (CF) baseplate. This structure is henceforth referred to as the sensor-spacer-baseplate-assembly (SSBA). This project will focus on the assembly of the SSBA only. The precision requirements of the SSBA are shown in Figure \ref{fig:ps(2s)_precision}. For the PS module, the sensors must align to within 40 mm measured at the sensor’s short edge. This corresponds to a rotational alignment tolerance of 0.8 mrad \cite{AutomatedAssembly_tutorial}.

\begin{figure}[ht]\centering
\includegraphics[width=0.4\linewidth]{Data/Introduction/PS(2S)_precision.png}
\caption{The precision requirements of the assembly of PS
and 2S modules.}
\label{fig:ps(2s)_precision}
\end{figure}
