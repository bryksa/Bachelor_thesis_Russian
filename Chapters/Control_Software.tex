\chapter{Программное обеспечение контроля сборки}

С целью контроля всей системы автоматизированной сборки было разработано специальное программное обеспечение (Qt~приложение) для ПК. Всё необходимое аппаратное обеспечение подключается по интерфейсу USB.

\section{Общая структура приложения}

Контроль подсистемами движения, наблюдения и отрицательного давления интегрированы в одно приложение, именуемого в дальнейшем PSAuto (от автоматизированной сборки PS модуля). Приложение написано на языке программирования C++ и использует Qt~фреймворк версии 4.8.7. Схематически интеграция трёх подсистем автоматизированной сборки изображена на Рисунке \ref{fig:general_app_structure}.

\begin{figure}[ht]\centering
\includegraphics[width=1\linewidth]{Data/Control_Software/Whole_system_diagram_(English).png}
\caption{Схематичческое изображение интеграции подсистемами движения, наблюдения и отрицательного давления с помощью Qt~приложение PSAuto.}
\label{fig:general_app_structure}
\end{figure}

\subsection{Шаблон проектирования Модель-Представление-Контроллер}

Структура приложения построено согласно шаблону проектирования Модель-Представление-Контроллер (Model-View-Controller (MVC)). В нём рассматриваются три основных типа объектов: объекты \emph{модели}, объекты \emph{представления} и объекты \emph{контроллера}. При разработке приложения важным шагом является выбор или создание пользовательских классов для объектов, которые попадают в одну из этих трех групп. Каждый из трех типов объектов отделен от других абстрактными границами. Шаблон определяет не только объекты ролей в приложении, но также определяет способ взаимодействия объектов друг с другом \cite{apple_MVC}. Ключевым моментом MVC является то, что объекты Представления и Контроллера зависят от объектов Модели, однако объекты Модели в свою очередь не зависит от них. Взаимодействие данных типов объектов схематически показано на Рисунке \ref{fig:mvc_general}.

\begin{figure}[ht]\centering
\includegraphics[width=0.7\linewidth]{Data/Control_Software/MVC_general.png}
\caption{Модель, Представление and Контроллер (MVC) относительно пользователя (классическая взаимодействие).}
\label{fig:mvc_general}
\end{figure}

На Рисунке \ref{fig:mvc_general} показано взаимодействие объектов классической архитектуры MVC. Существует различные варианты данной диаграммы с точки зрения взаимодействия между объектами MVC. Это зависит от типа приложения и его реализации. Например, в приложении PSAuto данная диаграмма будет иметь вид как на Рисунке \ref{fig:mvc_custom}.

\begin{figure}[ht]\centering
\includegraphics[width=0.7\linewidth]{Data/Control_Software/MVC_custom.png}
\caption{Модель, Представление and Контроллер (MVC) относительно пользователя (взаимодействие в приложении PSAuto).}
\label{fig:mvc_custom}
\end{figure}

По сравнению с классической структурой MVC как на рисунке \ref{fig: mvc_general}, Модель напрямую не информирует/обновляет Представление. Эта информация проходит через Контроллер, который действует как посредник между Моделью и Представлением. Контроллер часто отвечает за то, чтобы Представление имело доступ к объектам Модели, которые им нужно отображать, и выступает в качестве канала, через который объект Представления узнаёт об изменениях в Модели. Объекты Контроллера также могут выполнять задачи настройки приложения и управлять рабочими процессами других объектов \cite{apple_MVC}.

В зависимости от логики и требований приложения, Модель, Представление и Контроллер следуют определённым требованиям и обладают определёнными свойствами.

\emph{Модель} является центральным компонентом системы. Она напрямую работает с данными, определяя логику и правила работы приложения. Более того, как в случае с PSAuto, Модель ещё и регулирует взаимодействие между программным и аппаратным обеспечением. Например, в приложении PSAuto есть объект Модели \emph{ConradModel}, который отвечает за взаимодействие с релейной картой, контролирующей линии отрицательного давления системы. Данный объект полностью согласуется с шаблоном проектирования MVC как объект Модели, так как не зависит от других классов, однако другие классы зависят от него, используя его возможности для своего функционирования.

\emph{Представление} объединяет все объекты, отвечающие за графический интерфейс приложения: окна, вкладки, формы, поля, кнопки и т.п. Объекты представления отвечают за отображение информации, предоставляемой объектами модели, а также предоставляет пользователю вносить изменения в работы приложения (осуществлять контроль процесса сборки). Например, в приложении PSAuto одна из задач объекта \emph{AssemblyModuleAssembler} контролирует отображение информации о состоянии линий отрицательного давления в реальном времени, а также элементы их управления. Однако, он только отображает элементы контроля и передаёт действия пользователя объектам Контроллера, но сам не участвует в обработке этих действий.

\emph{Контроллер} в основном ответственный за приём и обработку поступающих данных, генерацию команд для Модели и Представления. Например, в приложении PSAuto объект \emph{ConradManager} предоставляет весь необходимый функционал для контроля линиями отрицательного давления и получения информации об их текущем статусе. Также стоит отметить, что очень часто не существует прямой связи между Моделью и Представлением, как показано на Рисунке \ref{fig:mvc_general}. Вместо этого Контроллер выступает в качестве промежуточного звена, передающего информацию между Моделью и Представлением.

\subsection{Библиотека OpenCV}

Множество алгоритмов, обеспечивающий функционал приложения PSAuto, реализуется при помощи библиотеки с открытым кодом~---~OpenCV (\textit{Open Source Computer Vision}). Это кроссплатформенная библиотека алгоритмов, в основном ориентированных на компьютерное зрение и обработку изображений в реальном времени (OpenCV -- от англ. Computer Vision). Данная библиотека была разработана в исследовательском центре Intel в Нижнем Новгороде (Россия). Она распространяется на условиях лицензии BSD и может быть свободна использована в академических целях.

OpenCV была специально ориентирована на высокую вычислительную эффективность и, как уже упоминалось, ориентированность на работу в реальном времени. Она полностью реализована на языке программирования C++ и способна использовать преимущества многоядерных процессорных систем. Также она поддерживает возможность автоматической оптимизации под архитектуру продуктов Intel с помощью библиотек \textit{Integrated Performance Primitives (IPP)}, которые состоят из низкоуровневых оптимизированных подпрограмм в различных алгоритмических областях~\cite{kaehler2016learning}.

Одна из основных задач библиотеки OpenCV -- предоставить простую и понятную в использовании инфраструктуру для работы с приложениями компьютерного зрения, что оказывает серьёзную поддержку в разработке достаточно сложных приложений данной области. Библиотека насчитывает более 500 функций, охватывающих множество областей компьютерного зрения и обработки изображений. Ниже приведены основные функции библиотеки OpenCV:

\begin{itemize}
\setlength\itemsep{-0.5em}
\item Ввод/вывод изображений и видео (открытие изображений из файла или непосредственное получение от камеры, вывод изображений/видеофайлов)
\item Операции линейной алгебры над векторами и матрицами (умножение, определители, собственные значения, сингулярное разложение).
\item Различные динамические структуры данных (списки, очереди, наборы, деревья, графики).
\item Основы обработки изображений (фильтрация, обнаружение краев, определение угла, выборка и интерполяция, преобразование цвета, морфологические операции, гистограммы и т.д.).
\item Структурный анализ (связанные компоненты, обработка контуров, дистанционное преобразование, совпадение шаблонов, преобразование Хафа, многоугольное приближение, линейная подгонка, эллиптическая подгонка, триангуляция Делоне).
\item Калибровка камеры (поиск и отслеживание шаблонов калибровки, калибровка, оценка фундаментальной матрицы, оценка гомографии, стереозапись).
\item Анализ движения (оптический поток, сегментация движения, отслеживание).
\item Распознавание объектов (метод собственных значений, метод скрытой Марковской Модели).
\item Основные элементы графического интерфейса (отображение изображения/видео, управление клавиатурой и мышью, полосы прокрутки).
\item Маркировка изображений (линии и другие фигуры, текстовые подписи).
\end{itemize}

Применение функционала библиотеки будет описано в следующих параграфах.

\section{Распознавание образов}

Распознавание образов является важной функцией всей автоматизированной системы сборки, в частности - подсистемы наблюдения. Она обеспечивает программное обеспечение информацией расположении платформы и компонентов модуля в пространстве и их планарной ориентации. С помощью этой информации программное обеспечение способно вычислить, куда перемещать каждый компонент модуля во время сборки. По определению, планарная ориентация -- это вращательная ориентация объекта в горизонтальной плоскости (XY в данном случае) \cite{AutomatedAssembly_tutorial}.

Для системы автоматизированной сборки планарная ориентация проводится в два этапа: независимое определение \emph{положения} и \emph{ориентации} каждого из четырёх маркеров на углах сенсора и совокупность положений четырёх маркеров, из которой вычисляется конечная точная ориентация всего сенсора. Это делается путем обработки изображений \emph{точных маркеров} в углах сенсоров (Рисунок \ref{fig:fiducial_marker}), полученных подсистемой наблюдения. Маркеры точно расположены относительно полосок/пикселей сенсора, следовательно, узнав точное положение маркеров, точное позиционирование полосок/пикселей гарантировано. В качестве входных данных алгоритм берет необработанные изображения и возвращает значения положения и ориентации маркеров, расположенных в углах сенсорах PS модуля. Алгоритм распознавания образов использует пакет библиотеки \emph{OpenCV}. Ниже перечислены этапы распознавание положения и ориентации сенсора:

\begin{enumerate}
\setlength\itemsep{-0.5em}
\item Предварительная обработка изображения, полученного с камеры.
\item Определение позиции и планарной ориентации маркера.
\item Определение приблизительного положения следующего маркера и/или конечное вычисление планарной ориентации сенсора.
\end{enumerate}

\begin{figure}[ht]\centering
\includegraphics[width=0.7\linewidth]{Data/Control_Software/Fiducial_marker.png}
\caption{Принятые за основу сравнения маркеры на углах сенсора PS модуля.}
\label{fig:fiducial_marker}
\end{figure}

\subsection{Предварительная обработка изображения, полученного с камеры}

Необработанное изображение с камеры сперва преобразуется из цветного в изображение в оттенках серого, что известно как "\emph{grayscaling}" (от англ. grayscale -- оттенки серого) в библиотеке OpenCV. Пиксели изображения в оттенках серого содержат информацию об интенсивности в виде одного числа от 0 до 255, характеризующего степень темноты оттенка серого, в отличие от информации об интенсивности и цвете, содержащейся в пикселях цветного изображения. Поскольку маркеры основаны не на цветах, а на простых формах (окружность, угол), то информация о цвете не является полезной и потому не учитывается. Вместо этого изображение в оттенках серого преобразуется в \emph{бинарное} изображение, у которого каждый пиксель либо белый, либо чёрный. Данная процедура получила название \emph{"Thresholding"}, от англ. threshold -- порог (пороговое значение). Операция Thresholding заключается в простом конвертировании каждого пикселя изображения в оттенках серого в белый (чёрный) пиксель, если его интенсивность выше(ниже)  заданного порогового значения. Данная операция призвана уменьшить различие между изображениями маркеров идентичных сенсоров из-за случайного \emph{шума}, вызванного пылью и случайными различиями на поверхностях маркеров сенсора. Примеры изображения маркера в оттенках серого и его бинарных изображений для различных пороговых значений показаны на Рисунке \ref{fig:threshold}. Оптимальное значение порогового значения зависит от общего уровня освещённости вокруг места проведения сборки и контраста маркера на поверхности сенсора. На данный момент, типичное значение порогового значения составляет около 90, что приблизительно соответствует освещению лаборатории лампами дневного света.

\begin{figure}[ht]\centering
\includegraphics[width=0.9\linewidth]{Data/Control_Software/Threshold.png}
\caption{Применение операции Thresolding на изображении маркера}
\label{fig:threshold}
\end{figure}

\subsection{Определение позиции и планарной ориентации маркера}

Положение и ориентация маркера на бинарном изображении определяется с использованием стандартным для такой ситуации методом обработки изображений, известным как \emph{сопоставление шаблонов}. Библиотека OpenCV предоставляет одноимённую функцию для выполнения таких задач. Метод находит ту часть исследуемого изображении, которая больше всего походит на шаблонное изображение. В случае автоматизированной сборки исследуемым изображением является преобразованное в бинарное изображение полученное с камеры, а шаблонным -- заранее подготовленное бинарное изображение маркера. 

Сопоставление шаблонов начинает с итеративного сопоставления шаблонного изображения в каждой точке исследуемого изображения и высчитывает определённое числовое значение, или метрику, показывающее степень схожести шаблонного изображение с частью исследуемого изображения, которая соответствует данной точки. В приложении PSAuto эта точка является левой-верхней точкой данной части исследуемого изображения. Библиотека OpenCV предоставляет множество вариантов вычисления значения этой метрики. Аналогичные результаты наблюдаются для каждой из возможных метрик, поэтому была выбрана наиболее скоростная по вычислительной сложности -- метрика, основанная на нормированной квадратичной разности интенсивностей налагающихся пикселей исследуемого изображения и шаблона:

\begin{center}
$R(x,y)=\dfrac{\sum_{x',y'}^{}(T(x',y')-I(x+x',y+y'))^{2}}{\sum_{x',y'}^{}\sqrt{\sum_{x',y'}^{}T(x',y')^{2}\cdot\sum_{x',y'}^{}I(x+x',y+y')^{2}}}$
\end{center}
где I означает \emph{исследуемое изображение}, T -- \emph{шаблонное изображение} and R -- \emph{результирующая метрика}.


Данная метрику можно найти под именем CV\_TM\_SQDIFF\_NORMED в библиотеке OpenCV. Точка исследуемого изображения, в которой метрика достигнет минимального значения показывает наиболее вероятное положение шаблонного изображения, маркера в случае автоматизированной сборки. На Рисунке \ref{fig:template_matching} изображен процесс определение положения маркера с помощью сопоставления шаблона и его результат с использованием маркера на пробном сенсоре. После нахождения наиболее вероятного положение маркера алгоритм выделяет его белым прямоугольником на исследуемом изображении. Наблюдаемое на Рисунке \ref{fig:template_matching} местоположение близко соответствует ожиданиям.

\begin{figure}[ht]\centering
\includegraphics[width=0.9\linewidth]{Data/Control_Software/Template_matching.png}
\caption{Методика сопоставления шаблонов проиллюстрирована на левом изображении. Красные стрелки указывают направления итеративного расчет метрики в каждой точке основного изображения. Результат работы алгоритма сопоставления шаблонов при использовании тестовых изображений показан на правом изображении. Наиболее вероятное расположение маркера в главном изображении обозначается красным прямоугольником.}
\label{fig:template_matching}
\end{figure}

Чтобы вычислить ориентацию маркера в плоскости, перпендикулярной оптической оси камеры, процедура сопоставления шаблонов итеративно повторяется с различными вращательными преобразованиями для разных углов (в установленных пределах с определённым шагом), применяемыми к части исследуемого изображения, где предварительно маркер уже был найден без информации о его ориентации на плоскости. В каждой итерации сохраняется минимальное значение метрики, а по окончании цикла строится график зависимости значения метрики от угла поворота исследуемого изображения. В силу симметричности формы маркера, данный график напоминает параболу, ветви которой направлены вверх, с минимум в точки того угла поворота исследуемого изображения, при котором шаблонное изображение лучше всего накладывается на исследуемое изображение. На Рисунке \ref{fig:template_rotation} схематично изображено определение планарной ориентации маркера, в правой части которого изображён график, соответствующий тесту выполненному с изображением сенсора, где он имел планарную ориентацию около 3.5 градуса. Соответственно на графике чётко виден минимум при $\approx$3.5 градусах. Большая точность измерение ориентации сенсора может быть достигнута путём оптимизации таких факторов, как окружающее освещение, фокусировка изображения и форма маркера. Наилучшая точность, достигнутая во время тестов, составила 0.025 градуса, что более чем достаточно для целей проекта, так значение ориентации, полученное таким способом используется лишь для перепроверки конечной ориентации сенсора.

\begin{figure}[ht]\centering
\includegraphics[width=0.9\linewidth]{Data/Control_Software/Template_rotation.png}
\caption{На левом изображении схематично изображена оценка планарной ориентации сенсора. Красные стрелки указывают направление вращательного преобразования, применяемого к основному изображению. График зависимости минимального значения метрики от углового преобразования показан на правом изображении. Наблюдается явный минимум при $\approx$3.5~градусах}
\label{fig:template_rotation}
\end{figure}

\subsection{Определение приблизительного положения следующего маркера и/или конечное вычисление планарной ориентации сенсора}

Процедура, описанная на шаге 2, повторяется для каждого угла сенсора. Значение планарной ориентации, определенные в каждом углу, используются для того, чтобы задать направление движения, необходимое для того, чтобы роборука автоматически переместилась к соседнему углу. Конечная позиция и ориентация датчика определяются с помощью критерия $\chi^{2}$ для четырех точек (x, y). Полученное таким образом значение ориентации сенсора сверяется с приближёнными значениями ориентации для маркера каждого из четырёх углов. Если они совпадают в пределах допустимой погрешности, то это значение ориентации сенсора может использоваться в дальнейшем.

Подробная UML диаграмма алгоритма распознавания образов в системе автоматизированной сборки продемонстрирована в Приложении~\ref{fig:uml_uml_pattern_recognition}.

\section{Функционал приложения}

Главное окно приложения PSAuto имеет несколько кнопок и флагов в верхней части и ряд вкладок ниже, включая следующие: Finder, Threshold, Assembly, Autofocus, Motion Manager и другие.

\begin{enumerate}

\item \emph{Finder.} От англ. finder -- искатель. Простая вкладка, единственной задачей которой является демонстрация последнего снятого снимка камерой. Также в этой вкладке это изображение можно сохранить.

\item \emph{Threshold.} От англ. threshold -- порог, граница. Результат операции Thresholding в высокой степени зависит от условий освещения, поэтому каждый новый тест требует калибровки значения применяемого порогового значения для максимально эффективной работы операции Thresholding. Данная вкладка специально создана для осуществления контроля над данным значением. С её помощью можно настраивать пороговое значение и получать мгновенный отклик в виде применение операции Thresholding к последнему снятому изображению с новым пороговым значением. Изображения до и после применения данной операции также демонстируются в левой части вкладки. Скриншот вкладки Threshold показан на Рисунке \ref{fig:threshold_screenshot}.

\begin{figure}[ht]\centering
\includegraphics[width=0.7\linewidth]{Data/Control_Software/Threshold_screenshot.png}
\caption{Скриншот вкладки Threshold приложения PSAuto.}
\label{fig:threshold_screenshot}
\end{figure}

\item \emph{Assembly.} От англ. assembly -- cборка. Это основная вкладка приложения (Рисунок \ref{fig:assembly_screenshot}). Она включает в себя четыре окна с изображениями слева, средства управления справа и статус двигательной установки в реальном времени. Левое-верхнее окно содержит последнее снятое изображение. Левое-нижнее окно -- бинарное изображение после применения операции Thresholding. Правое нижнее окно содержит шаблонное изображение маркера. Наконец, правое-верхнее окно содержит финальный график распределения метрики последней операции распознавания образов.

\begin{figure}[ht]\centering
\includegraphics[width=0.7\linewidth]{Data/Control_Software/Assembly_screenshot.png}
\caption{Скриншот вкладки Assembly приложения PSAuto.}
\label{fig:assembly_screenshot}
\end{figure}

В правой части вкладки расположен набор средств управления (перечисляя сверху вниз):

\begin{itemize}
\setlength\itemsep{-0.5em}
\item Две кнопки для перемещения устройства захвата в абсолютные или относительные координаты, которые записываются в поля справа от них, соответственно.
\item Набор радио-кнопок для выбора типа шаблона маркера и режим работы распознавания образов.
\item Набор радио-кнопок для контроля релейной картой линий отрицательного давления. Справа от каждой кнопки отображается текущий статус соответствующей линии.
\item Последний инструмент позволяет проводить набор тестов системы. Он состоит из нескольких форм для исходной информации для теста и, ожидаемо, кнопку начала теста. Тесты выполняются в полностью автоматическом режиме и могут длиться неограниченно долго.
\end{itemize}

\item \emph{Auto focus.} От англ. auto focus -- автофокус. Ещё один полезный инструмент приложения -- способность проводить автоматическую фокусировку камеры. Данная функция играет очень важную роль в работе процедуры распознавания образов, так как от качества её проведения непосредственно зависит качество снимаемых изображений. Данная вкладка позволяет осуществлять две основные функции: найти положение камеры с лучшим фокусом и переместить камеру в это положение. Чтобы найти положения сфокусированное положение камеры необходимо пройти ряд шагов вдоль оси Z двигательной установки. Чем меньше шаг, тем точнее будет определено сфокусированное положение камеры. Идея работы данного алгоритма схожа с определением ориентации маркеров в алгоритме распознавания образов. Для каждого шага высчитывается и сохраняется определённое значение метрики, характеризующее степень размытости изображения, для чего используется функция библиотеки OpenCV getImageBlur(). Таким образом, сравнивая значения метрик каждого шага, программное обеспечение способно найти положение камеры, при котором получаемое изображение имеет наименьшее размытие.

Данный алгоритм также может использоваться и с другой целью. В силу того, что камера постоянно зафиксирована на роборуке, существует возможность измерения относительного расстояния вдоль оси Z до объекта под камерой. Данное применение возможно, например, для измерения толщины слоя клеящего вещества собранного СРСП или контроля любых других метрических параметров модуля.

\item \emph{Motion Manager.} От англ. motion manager -- менеджер передвижений. Данная вкладка предоставляет контроль над всеми механическими движениями системы. Её функционал включает в себя:

\begin{itemize}
\setlength\itemsep{-0.5em}
\item Независимый контроль вращения и движения по всем трём осям.
\item Автоматическая калибровка.
\item Слежением за статусом всех элементов двигательной подсистемы в режиме реального времени, включающие в себя четыре шаговых двигателей (один на вращение и три на движение по вдоль трёх осей).
\end{itemize}
\end{enumerate}


