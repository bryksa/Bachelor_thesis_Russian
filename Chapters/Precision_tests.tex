\chapter{Precision estimation tests}
The automated assembly system has a number of properties in terms of precision:
\begin{enumerate}
\item Motion stage movement repeatability.
\item Image acquiring repeatability.
\item Precision of pattern recognition.
\item Possible movements of a sensor while picking them up and down with the vacuum pick up tool.
\end{enumerate}

In order to investigate this properties a series of tests were done.
\\Real sensors will be very thin (around 200~um). This fact makes them very fragile. Even though dummy sensors, which will be used for further experiments, is a bit thicker (around 300~um), they are still too fragile for the first tests, because the bottom surface of the pick up tool and the plane underneath testing samples are not yet parallel enough. That is why for the first pick up and down tests we used glass samples. They have the same dimensions and represent close enough the properties of silicon sensors. Moreover, they are much cheaper, so that in case of test failure (sample break) it will not be such a big problem as if silicon sample crashes. Despite all mentioned above, none of glass samples where crashed.
\\Even though we did not do the pick up test with silicon samples, there is still an opportunity to get some information of the pick up and down precision of the silicon samples without direct tests with them. By making a full range of tests with glass samples, we will be able to say how pick up and down influences the precision. Based on this results we will be able to approximately predict the precision of pick up and down tests with silicon samples. Later, when parallelness of the bottom surface of the pick up tool and samples will be provided, we will be able to confirm the results of the prediction.

\section{Pattern recognition precision tests}

For investigation of the pattern recognition precision the following tests were done. During these tests samples were not moved, so that the additional errors by vacuum pick up and down can be excluded.

\subsection{Pattern recognition on the painted corner of a glass dummy}
In the very first test we investigated the pattern recognition on the corner of the sample. Thin pieces of glass with a silver painted corner (Figure \ref{fig:painted_corner}) were used for the tests as an approximation of a silicon sensor. Silver painted corner was used as a marker for pattern recognition to be found in the acquired image.  

\begin{figure}[ht]\centering
\includegraphics[width=0.8\linewidth]{Data/Precision_tests/Painted_corner.png}
\caption{Glass sample with silver painted corner.}
\label{fig:painted_corner}
\end{figure}

The step-by-step outline of this test is listed below:
\begin{enumerate}
\setlength\itemsep{-0.5em}
\item Move to the image acquiring position.
\item Acquire image and run pattern recognition.
\item Move aside for 5~mm in all three axes.
\item Move to the image acquiring position.
\item Acquire image and run pattern recognition.
\item Save data of the current iteration and go to the next one.
\end{enumerate}

After each step software saves the difference between measured coordinates before and after moving the arm with the camera. The distributions of these values are showed in Figure \ref{fig:corner_x}, Figure \ref{fig:corner_y} and Figure \ref{fig:corner_theta} for X-axis, Y-axis and theta, respectively. The test had 100~iterations done in a row.

\begin{figure}[ht]\centering
\includegraphics[width=0.8\linewidth]{Data/Precision_tests/Corner_c_x.png}
\caption{Distribution of the difference between detected X coordinate of the master image before and after moving the arm in each iteration. $\Delta X \approx 1~um$. }
\label{fig:corner_x}
\end{figure}

\begin{figure}[ht]\centering
\includegraphics[width=0.8\linewidth]{Data/Precision_tests/Corner_c_y.png}
\caption{Distribution of the difference between detected Y coordinate of the master image before and after moving the arm in each iteration. $\Delta Y \approx 2~um$.}
\label{fig:corner_y}
\end{figure}

\begin{figure}[ht]\centering
\includegraphics[width=0.8\linewidth]{Data/Precision_tests/Corner_c_theta.png}
\caption{Distribution of the difference between detected angle orientation of the master image relatively to the acquired image before and after moving the arm in each iteration. $\Delta theta \approx 0.15 ~degree$. }
\label{fig:corner_theta}
\end{figure}

Looking at the Figures \ref{fig:corner_x} and \ref{fig:corner_y} one can see that the X, Y detection of the pattern recognition has enough good precision (\underline{1-2~um of error [?]}), while the theta detection results do not look so precise. There are several reasons of such behaviour. The main one them is shown on the Figure \ref{fig:corner_threshold}.

Silver painted surface is not flat in 10~um scale. Due to this roughness different amount of light reflects to the camera from different points along the painted surface. That is why the painted corner contains various shades of grey, which in some points are darker, than the table underneath the sample (background). All these result into the picture one can see in the Figure \ref{fig:corner_threshold}. This kind of pictures has random distribution of dark areas on it. That is why the pattern recognition algorithm has such error while comparing two pictures (master template and acquired image) with random distribution of black areas. Moreover, this kind of tests lasts around one hour, which is long enough for the sun to change the ambient light in the laboratory. Even though all reasonably possible measures were done to prevent such effect, the acquired image is very sensitive for light. For example, the effect of the sun light can results in the threshold variation for 20 units (the color depth is 256) even with covered window in the laboratory.

\begin{figure}[ht]\centering
\includegraphics[width=0.8\linewidth]{Data/Precision_tests/Corner_thresholded.png}
\caption{The view of the corner after applying the Threshold.}
\label{fig:corner_threshold}
\end{figure}

\subsection{Pattern recognition on the marker of the dummy sensor}

The same test, but with dummy silicon sensor and real marker on it, was done. Before the test marker was aligned as much close to zero degrees as possible. At the Figure \ref{fig:thresholded_marker} one can see that the edge of the marker after applying Threshold is almost perfect (+/- one pixel). This fact itself is already a proof that Threshold step of pattern recognition is feasible.

\begin{figure}[ht]\centering
\includegraphics[width=0.6\linewidth]{Data/Precision_tests/Thresholded_marker.png}
\caption{Sensor marker after applying Threshold.}
\label{fig:thresholded_marker}
\end{figure}

The Distribution of X, Y coordinates and theta shows better results than with painted corner, which was expected. For X and Y it is less than a micron, which is already at the limit of camera resolution. $\Delta theta$ is one order of magnitude better than with painted corner --- $\approx 0.02  degree$.

A screenshot of the application during the test is shown on the Figure \ref{fig:zero_peak}. On the pattern recognition curve one can see that exactly at 0 degree there is an unexpected short upward shot. This peak is not a fluctuation and it is not consist of only one point in the plot. As the scale increases, more points form this peak. It was not observed in the previous test just because the theta step was one order of magnitude bigger.

\begin{figure}[ht]\centering
\includegraphics[width=0.8\linewidth]{Data/Precision_tests/Upward_shot.png}
\caption{Screenshot of application during precision estimation test with dummy silicon sensor and marker on it.}
\label{fig:zero_peak}
\end{figure}

\section{Pick-up and -down precision tests}

Next set of tests were oriented to explore the effect of picking-up/down on the precision of the system. In other words, will this process move the sample or not. Due to the risk of breaking fragile silicon sample, we used a glass one.

\subsection{Pick-up and -down precision tests without assembly platform}

The first pick-up/down test was done without assembly platform. The step-by-step outline of this test is listed below:
\begin{enumerate}
\setlength\itemsep{-0.5em}
\item Move to the image acquiring position.
\item Acquire image and run pattern recognition.
\item Move to pre-pickup position (!).
\item Move to pickup position.
\item Turn negative vacuum on.
\item Move up.
\item Move down.
\item Release vacuum.
\item Move down.
\item Move to pre-pickup position (!).
\item Move to the image acquiring position.
\item Acquire image and run pattern recognition.
\item Save data of the current iteration and go to the next one.
\end{enumerate}

One can notice the step of moving first to pre-pickup position before going to the pick-up position. This step is essential. The motion stage provides equal speed in all three axes, so when it receive a command to move to some position it starts to move with equal speed in each of three directions towards the destination simultaneously. As soon as destination in one axis is reached, it obviously stops moving in this axis while moving in other axes is going on. Therefore there is an unlike situation when the robotic arm reaches the sample in Z-axis while X and Y axes would still keep moving. To prevent such situation we added the step of moving to pre-pickup position.

The test showed significant movement of the sample

\subsection{Pick-up and -down precision tests with assembly platform}

