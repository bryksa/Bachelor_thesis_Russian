\chapter{Тестирование системы и сборка первого прототипа}

Система автоматизированной сборки обладает рядом свойств с точки зрения точность сборки:
\begin{enumerate}
\setlength\itemsep{-0.5em}
\item Повторяемость передвижений двигательной установки.
\item Повторяемость съёмки изображений.
\item Точность процедуры распознавания образов.
\item Возможные движения сенсора во время его захвата и отпускания с помощью системы отрицательного давления.
\end{enumerate}

С целью изучения данных свойств был проведён ряд тестов системы.

Настоящие сенсоры будут иметь очень малую толщину (200~мкм), что делает их очень хрупкими. Даже несмотря на то, что для тестов могут быть использованы фиктивные сенсоры, толщина которых составляет 300~мкм, они всё ещё слишком хрупки для первых тестов, во время которых вероятность нештатных ситуаций заметно выше. По этой причине для первых тестов захвата использовались использовались образцы из стекла. Они имеют такие-же размеры и достаточно близки по механическим свойствам с кремниевыми сенсорами. Более того, они заметно дешевле и, следовательно, в случае неудачного теста потеря стеклянного образца не так критична, как кремниевого фиктивного сенсора. Несмотря на всё, сказанное выше, ни один из стеклянных образцов не был разрушен во время тестов и тестовых сборок.

Хотя тесты захвата кремниевых сенсоров не проводились в виду рисков, возможность получения некоторой информации об их поведение в данном тесте можно спрогнозировать без непосредственного проведения теста. Проведя полный набор тестов со стеклянными образцами и проанализировав их результаты, можно будет сделать выводы о том, как процесс захвата и выпуска влияет на точность/позиционирование образца. Позднее, после точной калибровки параллельности нижней поверхности устройства захвата относительно поверхности сборочной платформы и проведения необходимых тестов программного обеспечения, можно будет уточнить предсказанные результаты с гораздо меньшими рисками неудачного исхода.

\section{Тест точности процедуры распознавания образов}

С целью исследования точности процедуры распознавания образов ряд следующих тестов был проведён. В ходе этих тестов исследуемые образцы (и стеклянные, и кремниевые) оставались неподвижными, поэтому ошибки, вносимые нежелательным ожидаемым смещением образца во время процедуры захвата, исключаются.

\subsection{Распознавание образов на угле стеклянного образца}

В самом первом тесте точности распознавания образов в качестве маркеров стеклянных образцов просто использовались их углы. Для теста использовались тонкие образцы стекла с углами, покрашенными серебряной краской для лучшего контраста на бинарном изображении (Figure \ref{fig:painted_corner}).

\begin{figure}[ht]\centering
\includegraphics[width=0.8\linewidth]{Data/Precision_tests/Painted_corner.png}
\caption{Образцы стекла с углами, покрашенными серебряной краской.}
\label{fig:painted_corner}
\end{figure}

Пошаговое описание теста приведено ниже:

\begin{enumerate}
\setlength\itemsep{-0.5em}
\item Передвинуться в позицию для съёмки изображения маркера.
\item Сделать снимок и провести процедуру распознавания образов.
\item Передвинуться на 5~мм по всем осям
\item Вернуться в позицию для съёмки изображения.
\item Сделать снимок и провести процедуру распознавания образов.
\item Сохранить данные данной итерации и перейти к следующей (к шагу 1).
\end{enumerate}

После каждой итерации приложение сохраняет разницу между измеренными координатами маркера до и после перемещения роборуки. Итоговое распределение данных величин изображено на Рисунках \ref{fig:corner_x}, \ref{fig:corner_y} и \ref{fig:corner_theta} для оси X, Y и $\alpha$, соответственно. Тест состоял из 100 итераций, сделанных подряд.

\begin{figure}[ht]\centering
\includegraphics[width=0.8\linewidth]{Data/Precision_tests/Corner_c_x.png}
\caption{Распределение разности по координате X до и после премещения роборуки. $\Delta X \approx 1~um$. }
\label{fig:corner_x}
\end{figure}

\begin{figure}[ht]\centering
\includegraphics[width=0.8\linewidth]{Data/Precision_tests/Corner_c_y.png}
\caption{Распределение разности по координате Y до и после премещения роборуки. $\Delta Y \approx 2~um$.}
\label{fig:corner_y}
\end{figure}

\begin{figure}[ht]\centering
\includegraphics[width=0.8\linewidth]{Data/Precision_tests/Corner_c_theta.png}
\caption{Распределение разности значения определённого угла поворота между исследуемым изображением относительно шаблонного изображения до и после перемещения роборуки. $\Delta theta \approx 0.15 ~degree$. }
\label{fig:corner_theta}
\end{figure}

На Рисунках \ref{fig:corner_x} и \ref{fig:corner_y} можно видеть, что повторяемость измерения координат X, Y находится в пределах 1-2~мкм, в то время как измерения угла $\alpha$ показывает гораздо меньшую точность. На то есть несколько причин. Основная из них показана на Рисунке \ref{fig:corner_threshold}.

Поверхность серебряной краски не является достаточно ровной в масштабах микрометров. Из-за этих неровностей свет неравномерно отражается от поверхности, что приводит к тому, что на изображении угла в оттенках серого покрашенная поверхность в некоторых точках темнее поверхности (фон) под образцом. Результат данного эффекта можно наблюдать на Рисунке \ref{fig:corner_threshold}. Изображения такого рода имеют случайной распределение тёмных областей. Именно поэтому процесс распознавания образов выдают такие неточности сравнивая исследуемое изображение со \emph{случайным распределением} тёмных областей и шаблонного изображения с \emph{постоянным распределением} тёмных областей. Более того, для набора достаточного числа статистических данных тесты такого рода содержат большое количество итераций (100 в данном случае), что соответствует нескольким часам работы установки. Этого времени достаточно для того, чтобы естественное освещение в лаборатории изменилось. Несмотря на принятые меры изоляции установки от естественного освещения, оптимальное пороговое значение операции Thresholding изменилось на 20 единиц (глубина цвета изображения в оттенках серого -- 256 единиц) за время эксперимента.

\begin{figure}[ht]\centering
\includegraphics[width=0.8\linewidth]{Data/Precision_tests/Corner_thresholded.png}
\caption{Бинарное изображение покрашенного угла стеклянного образца после применения процедуры Thresholding.}
\label{fig:corner_threshold}
\end{figure}

\subsection{Распознавание образов на маркере фиктивного сенсора}

Аналогичный предыдущему тест был проведён с фиктивным сенсором, имеющем действительный маркер. Для удобства проведения теста перед его началом сенсор был ориентирован как можно ближе к положению, соответствующему $\alpha=0$. На Рисунке \ref{fig:thresholded_marker} показан маркер фиктивного сенсора. После применения операции Thresholding граница маркера практически идеальна (+/- один пиксель). Данный факт подтверждает обоснованность применения операции Thresholding ещё до проведения тестов.

\begin{figure}[ht]\centering
\includegraphics[width=0.6\linewidth]{Data/Precision_tests/Thresholded_marker.png}
\caption{Маркер фиктивного сенсора после проведения операции Thresholding.}
\label{fig:thresholded_marker}
\end{figure}

Распределение величин X,Y и $\alpha$ показывает намного лучшие результаты, чем с у покрашенного серебряной краской угла, что вполне ожидаемо. Для координат X и Y точность составляет меньше микрона, что уже граничит с пределом разрешающей способности камеры. $\Delta theta$  в свою очередь стала на один порядок точнее, чем у окрашенного угла стеклянного образца -- $\approx 0.02  degree$.

Скриншот работы приложения во время теста изображён на Рисунке \ref{fig:zero_peak}. На графике распознавания образов в точке, соответствующей 0~градусов, можно заметить неожиданный короткий пик, направленный вверх. Данный пик не является флуктуацией и не состоит только из одной точки на графике. С увеличением масштаба графика, всё больше и больше точек начинают принимать участие в его формировании. Он не наблюдался в предыдущих тестах, так как шаг $\alpha$ был на порядок выше.

\begin{figure}[ht]\centering
\includegraphics[width=0.8\linewidth]{Data/Precision_tests/Upward_shot.png}
\caption{Скриншот работы приложения во время проведение теста точности и повторяемости с фиктивным кремниевым сенсором.}
\label{fig:zero_peak}
\end{figure}

\section{Тест захвата образца}

Следующий ряд тестов был ориентирован на исследование эффекта нежелательного эффекта механического смещения образца во время его захвата и выпускания. В виду высокого риска механического повреждения сенсора использовались стеклянные образцы.

\subsection{Тест захвата образца без помощи сборочной платформы}

Самый первый тест захвата/выпускания был выполнен без помощи сборочной платформы. Его пошаговое описание приведено ниже:

\begin{enumerate}
\setlength\itemsep{-0.5em}
\item Передвинуться в позицию съёмки изображения маркера.
\item Сделать снимок и провести процедуру.
\item Передвинуться в предварительную позицию захвата.
\item Передвинуться в позицию захвата.
\item Подать отрицательное давление на устройство захвата.
\item Передвинуться вверх на 5~см.
\item Передвинуться обратно вниз на 5~см.
\item Отключить подачу отрицательного давления на устройство захвата.
\item Передвинуться в предварительную позицию захвата.
\item Передвинуться в позицию съёмки изображения маркера.
\item Сделать снимок и провести процедуру
\item Сохранить данные текущей итерации и перейти к следующей (к шагу 1).
\end{enumerate}

Перемещение сперва в предварительную позицию захвата перед непосредственным переходом к позиции захвата очень важно. Двигательная установка обеспечивает равную скорость движения по всем трём осям, поэтому, когда она получает команду для перемещения в какое-либо положение, то начинает двигаться одновременно с равной скоростью по всем трём осям к месту назначения. Как только необходимая координата на одной оси достигнута, двигательная установка, очевидно, перестает двигаться по этой оси, а движение по другим осям продолжается. Поэтому может возникнуть нежелательная ситуация, когда роботизированная рука достигает образца по оси Z, в то время как оси X и Y все равно будут двигаться, что может привести к повреждению или даже уничтожению образца. Чтобы предотвратить такую ситуацию, был добавлен шаг перехода в в предварительную позицию захвата, которая отличается от позиции захвата только координатой Z -- она на 5~см выше. Таким образом, из позиции предварительного захвата роборука перемещается строго вниз и безопасно касается образца для его захвата.

Результаты теста показали смещение образца, которое можно было заметить даже невооруженным глазом. Чтобы свести к минимуму это движение, было решено провести "тест касанием" - те же этапы, что и у теста захвата, но без подачи отрицательного давления. Такой тип теста может показать вклад касания к перемещению образца в тесте захвата. Результаты этого теста были очень похожи на результаты предыдущего теста, что означает, что непосредственно касание сенсора вносит наибольший вклад в смещение образца во время процедуры захвата. Интересным фактом этого движения является его характер. На рисунке \ref{fig:touch_move} можно увидеть зависимость движения образца по осям X и Y относительно номера итерации в тесте. Эти графики показывают, что движение не является случайным, а является более-менее постоянным как по значению, так и по направлению.

\begin{figure}[ht]\centering
\includegraphics[width=1\linewidth]{Data/Precision_tests/XY_touch_movement.png}
\caption{Движения образца по осям X и Y относительно номера итерации в тесте касанием.}
\label{fig:touch_move}
\end{figure}

Наиболее вероятной причиной такого поведения сенсора является антипараллельность поверхности устройства захвата и поверхности стола, на котором располагались образцы в данном тесте. Этот факт хорошо соответствует постоянному смещению образца в одном направлении. К сожалению, очень сложно выровнять эти поверхности достаточно параллельно, чтобы движение образца было незначительным, так как в любом случае положение образца будет представлять из себя неустойчивое равновесие. Другой, более простой и эффективный, способ избежать этого движения - использовать сборочную платформу, которая может удерживать образец отрицательным, не допуская какого либо их движения.

\subsection{Тест захвата образца с помощью сборочной платформой}

Пошаговое описание тест захвата образца с помощью сборочной платформой очень похож на тест без неё. Различием между ними является постоянная фиксация образца отрицательным давлением снизу и его отпускание только после включения подачи отрицательного давления устройством захвата. Другим словами происходит процесс "передачи" образца сборочной платформой устройству захвата. 

\begin{enumerate}
\setlength\itemsep{-0.5em}
\item Передвинуться в позицию съёмки изображения маркера.
\item Сделать снимок и провести процедуру.
\item Передвинуться в предварительную позицию захвата.
\item Передвинуться в позицию захвата.
\item Подать отрицательное давление на устройство захвата.
\item Отключить подачу отрицательного давление на сборочную платформу.
\item Передвинуться вверх на 5~см.
\item Передвинуться обратно вниз на 5~см.
\item Подать отрицательное давление на сборочную платформу.
\item Отключить подачу отрицательного давления на устройство захвата.
\item Передвинуться в предварительную позицию захвата.
\item Передвинуться в позицию съёмки изображения маркера.
\item Сделать снимок и провести процедуру
\item Сохранить данные текущей итерации и перейти к следующей (к шагу 1).
\end{enumerate}

Распределение разности между измеренной координатой X маркера до и после перемещения роборуки показано на Рисунке \ref{fig:platform_distribution}.

\begin{figure}[ht]\centering
\includegraphics[width=0.8\linewidth]{Data/Precision_tests/Platform_x_distrib.png}
\caption{Распределение разности между измеренной координатой X маркера до и после перемещения роборуки. $\Delta X \approx 4~um$. }
\label{fig:platform_distribution}
\end{figure}

Как можно видеть на Рисунке \ref{fig:platform_distribution}, $\Delta X \approx 4~um$, что очень близко к пределу процедуры распознавания образов ($\approx 1~um$) без касания образцы. Более того, Абсолютные значения измеренных координат X и Y показывают отсутствие какой либо зависимости от номера итерации, как в тесте без сборочной платформы. Принимая во внимания полученные результаты, можно утверждать, что сборочная платформа способна фиксировать образцы достаточно плотно, чтобы пренебречь их нежелательными механическими смещениями.

\section{Сборка первого прототипа}

После всех выше упомянутых тестов было решено провести сборку первого прототипа модуля. Для проверки точности его сборки было принято решение использовать упрощенный алгоритм сборки, который обеспечивал бы максимальную точность только на одном из четырёх углов. Схематическое изображение первого прототипа модуля показано на рисунке \ref{fig:module_prototype}.

\begin{figure}[ht]\centering
\includegraphics[width=0.8\linewidth]{Data/Precision_tests/Module_prototype.png}
\caption{Схематическое изображение первого прототипа модуля.}
\label{fig:module_prototype}
\end{figure}

Пошаговое описание упрощённого процесса сборки описано ниже:

\begin{enumerate}
\setlength\itemsep{-0.5em}
\item \textit{Подготовить верхний стеклянный образец.} Поместить верхний образец на сборочную платформу и зафиксируйте его с помощью отрицательного давления. С помощью процедуры распознавания образов определить его ориентацию в пространстве. Так необходимая точность определения планарной ориентации не обеспечивается путём его определения по одному маркеру, то для уточнения этого значения использовался следующий метод. Переместить камеру на 5 см вдоль края образца и измерьте смещение края перпендикулярно движению. Пройденное расстояние и смещение образуют два катета прямоугольного треугольника, зная которые можно вычислить недостающий угол поворота между кромкой образца и осью X двигательной установки. Далее следует повернуть платформу на этот угол. Другими словами, сперва -- грубая оценка угла ориентации образца с помощью процедуры распознавания образов, после -- точное вычисления угла по теореме Пифагора. После того, как образец правильно сориентирован, необходимо сохранить координаты опорного угла образца и провести процедуру его захвата и подъёма.
\item \textit{Подготовить алюминиевые распорки} Повернуть сборочную платформу на 90~градусов. Поместить алюминиевые распорки в выемки сборочной платформы, зафиксировать их отрицательным давлением и сориентировать их параллельно оси X двигательной установки. Данная процедура проделывается аналогично верхнему образцу. Однако стоит упомянуть, что достижение высокой точности в ориентации распорок не настолько важно, как для сенсоров в модуле. Далее следует сохранить координаты опорного угла распорок. Сравнивая эти координаты с координатами опорного угла верхнего образца, переместить роборуку в плоскости XoY для их совмещения.
\item \textit{Произвести склеивание алюминиевых распорок с верхним образцом.} Нанести клеящее вещество на распорки и переместить вниз роборуку с прикреплённым к ней верхним образцом. Очень важно переместиться вниз на корректное расстояние. Данная тема была подробнее обсуждена в Главе 2. Далее необходимо выждать время высыхания быстросохнущего клеящего вещества (около 15~минут) и поднять полученную структуру (две алюминиевые распорки + верхний образец) со сборочной платформы, предварительно выключив подачу отрицательного давления в платформе.
\item \textit{Подготовить нижний стеклянный образец.} Повернуть сборочную платформу на 90~градусов в противоположном направлении. Поместить нижний образец на сборочную платформу и зафиксировать его отрицательным давлением. По примеру верхнего образца провести ориентирование нижнего образца параллельно оси X двигательной установки. Измерить координаты опорного угла нижнего образца и сопоставить их с координатами верхнего. Переместить роборуку так, чтобы они совмещались.
\item \textit{Финальное склеивание.} Нанести клеящее вещество на нижний образец и переместить роборуку вниз на корректную дистанцию, учитывая толщины распорок, стеклянных образцов и желаемого слоя клеящего вещества. После высыхания вещества процесс автоматизированной сборки можно считать оконченным.
\end{enumerate}

Собранный прототип изображён на Рисунке \ref{fig:module_prototype}.

\begin{figure}[ht]\centering
\includegraphics[width=1\linewidth]{Data/Precision_tests/Prototype_photo.png}
\caption{Фото первого собранного прототипа PS модуля.}
\label{fig:module_prototype}
\end{figure}

Использованный упрощённый алгоритм автоматизированной сборки рассчитан на точное позиционирование только одного угла прототипа. Качество собранного прототипа выглядит многообещающим. Оно удовлетворяет всем требованиям. На Рисунке \ref{fig:prototype_macro} изображено фото опорного угла под микроскопом. Параметры угла с других направлений выглядят схоже, имея приблизительно одинаковую точность.

\begin{figure}[ht]\centering
\includegraphics[width=1\linewidth]{Data/Precision_tests/Top_sensor_X_view.png}
\caption{Фото опорного угла собранного прототипа под микроскопом. Несогласованность компонентов находится в пределах 20~мкм.}
\label{fig:prototype_macro}
\end{figure}

Собранный прототип подтвердил осуществимость системы автоматизированной сборки и показал её возможности, а также и моменты, на которые стоит обратить внимание в дальнейшем. Несмотря на то, что собранный прототип в качестве маркеров использовал окрашенные углы, а не точную литографию, как настоящие и фиктивные сенсоры, он показал очень хорошие результаты в плане точности сборки.
